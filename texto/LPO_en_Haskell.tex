% LPO_en_Haskell.tex
% Lógica de primer orden (LPO) en Haskell.
% Eduardo Paluzo
% Sevilla, 16 de julio de 2016
% =============================================================================

\documentclass[a4paper,12pt,twoside]{book}

%%%%%%%%%%%%%%%%%%%%%%%%%%%%%%%%%%%%%%%%%%%%%%%%%%%%%%%%%%%%%%%%%%%%%%%%%%%%%%
%%  Paquetes adicionales                                                   %%
%%%%%%%%%%%%%%%%%%%%%%%%%%%%%%%%%%%%%%%%%%%%%%%%%%%%%%%%%%%%%%%%%%%%%%%%%%%%%%

\usepackage[utf8x]{inputenc}       % Acentos de UTF8
\usepackage[spanish]{babel}        % Castellanización.
\usepackage[T1]{fontenc}           % Codificación T1 con European Computer
                                   % Modern.  
\usepackage{graphicx}
\usepackage{fancyvrb}              % Verbatim extendido
\usepackage{makeidx}               % Índice
\usepackage{amsmath}               % AMS LaTeX
\usepackage{amsthm} 
\usepackage{latexsym}
\usepackage[colorinlistoftodos
           , backgroundcolor = yellow
           , textwidth = 4cm
           , shadow
           , spanish]{todonotes}
% Fuentes
\usepackage{mathpazo}              % Fuentes semejante a palatino
\usepackage[scaled=.90]{helvet}
\usepackage{cmtt}
\renewcommand{\ttdefault}{cmtt}
\usepackage{a4wide}
% \usepackage{xmpincl}               % Incluye metadato de licencia CC.

% Tikz
\usepackage{tkz-berge}
\usetikzlibrary{shapes,trees}

% Desactivar <,> cuando se hacen dibujos con tikz.
\tikzset{
every picture/.append style={
  execute at begin picture={\deactivatequoting},
  execute at end picture={\activatequoting}
  }
}

% Márgenes
\usepackage[margin=1in]{geometry}

% Control de espacios en la tabla de contenidos:
\usepackage[titles]{tocloft}
\setlength{\cftbeforechapskip}{2ex}
\setlength{\cftbeforesecskip}{0.5ex}
\setlength{\cftsecnumwidth}{12mm}
\setlength{\cftsubsecindent}{18mm}

% Control de listas
% Elimina espacio entre item y párrafo y coloca item en el margen izquierdo
% \usepackage{enumitem}
% \setlist[enumerate,itemize]{noitemsep, nolistsep, leftmargin=*}

\usepackage{minitoc}

% Doble espacio entre líneas
\usepackage{setspace}
\onehalfspacing


% \linespread{1.05}                % Distancia entre líneas
\setlength{\parindent}{5mm}        % Indentación de comienzo de párrafo
\deactivatetilden                  % Elima uso de ~ para la eñe
\raggedbottom                      % No ajusta los espacios verticales.

\usepackage[%
  pdftex,
  pdfauthor={Eduardo Paluzo},%
  pdftitle={LPO en Haskell},%
  pdfstartview=FitH,%
  bookmarks=false,%
  colorlinks=true,%
  urlcolor=blue,%
  unicode=true]{hyperref}      

\usepackage{tikz}


%%%%%%%%%%%%%%%%%%%%%%%%%%%%%%%%%%%%%%%%%%%%%%%%%%%%%%%%%%%%%%%%%%%%%%%%%%%%%%
%%  Cabeceras                                                              %%
%%%%%%%%%%%%%%%%%%%%%%%%%%%%%%%%%%%%%%%%%%%%%%%%%%%%%%%%%%%%%%%%%%%%%%%%%%%%%%

\usepackage{fancyhdr}

\addtolength{\headheight}{\baselineskip}

\pagestyle{fancy}

\cfoot{}

\fancyhead{}
\fancyhead[RE]{\bfseries \nouppercase{\leftmark}}
\fancyhead[LO]{\bfseries \nouppercase{\rightmark}}
\fancyhead[LE,RO]{\bfseries \thepage}

%%%%%%%%%%%%%%%%%%%%%%%%%%%%%%%%%%%%%%%%%%%%%%%%%%%%%%%%%%%%%%%%%%%%%%%%%%%%%%
%%  Definiciones                                                           %%
%%%%%%%%%%%%%%%%%%%%%%%%%%%%%%%%%%%%%%%%%%%%%%%%%%%%%%%%%%%%%%%%%%%%%%%%%%%%%%

\input definiciones
\def\mtctitle{Contenido}

%%%%%%%%%%%%%%%%%%%%%%%%%%%%%%%%%%%%%%%%%%%%%%%%%%%%%%%%%%%%%%%%%%%%%%%%%%%%%%
%%  Título                                                                 %%
%%%%%%%%%%%%%%%%%%%%%%%%%%%%%%%%%%%%%%%%%%%%%%%%%%%%%%%%%%%%%%%%%%%%%%%%%%%%%%

\title{\Huge Lógica de primer orden en Haskell}
\author{Eduardo Paluzo}
\date{\vfill \hrule \vspace*{2mm}
  \begin{tabular}{l}
      \href{http://www.cs.us.es/glc}
           {Grupo de Lógica Computacional} \\
      \href{http://www.cs.us.es}
           {Dpto. de Ciencias de la Computación e Inteligencia Artificial} \\
      \href{http://www.us.es}
           {Universidad de Sevilla}  \\
      Sevilla, 16 de junio de 2016 (Versión de \today)
  \end{tabular}\hfill\mbox{}}




%%%%%%%%%%%%%%%%%%%%%%%%%%%%%%%%%%%%%%%%%%%%%%%%%%%%%%%%%%%%%%%%%%%%%%%%%%%%%%%
%%  Construcción del índice                                                 %%
%%%%%%%%%%%%%%%%%%%%%%%%%%%%%%%%%%%%%%%%%%%%%%%%%%%%%%%%%%%%%%%%%%%%%%%%%%%%%%%

\makeindex

%%%%%%%%%%%%%%%%%%%%%%%%%%%%%%%%%%%%%%%%%%%%%%%%%%%%%%%%%%%%%%%%%%%%%%%%%%%%%%
%%  Documento                                                              %%
%%%%%%%%%%%%%%%%%%%%%%%%%%%%%%%%%%%%%%%%%%%%%%%%%%%%%%%%%%%%%%%%%%%%%%%%%%%%%%

% \includeonly{Introduccion}

% \includexmp{licencia}

\begin{document}

\dominitoc


\maketitle
\newpage

\comentario{La portada debe de ser análoga a la de
  \href{http://bit.ly/2a8PGz7}{Dani}.}

\input{licencia/licenciaCC}
\newpage 

\tableofcontents
\newpage

\chapter*{Introducción}
\addcontentsline{toc}{chapter}{Introducción}

El objetivo del trabajo es la implementación de los algoritmos de la lógica de
primer orden en Haskell. Consta de dos partes:

\begin{itemize}
\item La primera parte consiste en la adaptación de los programas del libro de
  J. van Eijck 
  \href{http://citeseerx.ist.psu.edu/viewdoc/download?doi=10.1.1.467.1441&rep=rep1&type=pdf}
       {Computational semantics and type theory}\
  \footnote{\url{http://citeseerx.ist.psu.edu/viewdoc/download?doi=10.1.1.467.1441&rep=rep1&type=pdf}}
  (\cite{Eijck-03}) y su correspondiente teoría.
\item En la segunda parte se programan en Haskell los algoritmos de la lógica
  de primer orden estudiados en la asignatura de 
  \href{https://www.cs.us.es/~jalonso/cursos/lmf-15}
       {Lógica matemática y fundamentos}\
  \footnote{\url{{https://www.cs.us.es/~jalonso/cursos/lmf-15}}}
  (\cite{Alonso-15a}).
\end{itemize}

%%% Local Variables: 
%%% mode: latex
%%% TeX-master: "LPO_en_Haskell"
%%% End: 


\chapter{Programación funcional con Haskell}

En este capítulo se hace una breve introducción a la programación funcional en
Haskell suficiente para entender su aplicación en los siguientes
capítulos. Para una introducción más amplia se pueden consultar los apuntes de
la asignatura de Informática de 1º del Grado en Matemáticas
(\cite{Alonso-15b}). También se puede emplear como lectura complementaria, y se
ha empleado para algunas definiciones del trabajo (\cite{YerPal-90})

El contenido de este capítulo se encuentra en el módulo \texttt{PFH} 
\entrada{PFH}.

%%% Local Variables:
%%% mode: latex
%%% TeX-master: "LPO_en_Haskell"
%%% End:

\chapter{Sintaxis y semántica de la lógica de primer orden}

El \textbf{lenguaje de la lógica de primer orden} está compuesto por

\begin{enumerate}
\item Variables proposicionales $p,q,r,\dots$
\item Conectivas lógicas:
  \begin{center}
   \begin{tabular}{| l | l |}
     \hline
      $\neg$   & \text{Negación} \\ \hline
      $\vee$   & \text{Disyunción} \\ \hline
      $\wedge$ & \text{Conjunción} \\ \hline
      $\rightarrow$ & \text{Condicional} \\ \hline
      $\leftrightarrow$ & \text{Bicondicional}\\
     \hline
   \end{tabular}
 \end{center}
\item Símbolos auxiliares $"(",")"$
\item Cuantificadores: $\forall$ (Universal) y $\exists$ (Existencial)
\item Símbolo de igualdad: $=$
\item Constantes: $a,b,\dots,a_1,a_2,\dots$
\item Símbolos de predicado : $P,Q,R,\dots $
\item Símbolos de función : $f,g,h\dots $
\end{enumerate}

\entrada{Modelo}.

% \chapter{Deducción natural}

\comentario{Eliminar este capítulo.}

En este capítulo se pretende implementar la deducción natural de la lógica de
primer orden en Haskell. El contenido de este capítulo se encuentra en el
módulo \texttt{DNH}.  

\entrada{DNH}

\chapter{El método de los tableros semánticos}
\label{sec:tableros}

Este capítulo pretende aplicar métodos de tableros para la demostración
de teoremas en lógica de predicados. El contenido de este capítulo
se encuentra en el módulo \texttt{PTLP}.

\entrada{PTLP}

%%% Local Variables:
%%% mode: latex
%%% TeX-master: "LPO_en_Haskell"
%%% End:

\chapter{Modelos de Herbrand}

En este capítulo se pretende formalizar los modelos de Herbrand.
Herbrand propuso un método constructivo para generar interpretaciones
de fórmulas o conjuntos de fórmulas.

\entrada{Herbrand}
\chapter{Resolución en lógica de primer orden}

En este capítulo se introducirá la resolución en la lógica de primer orden
y se implementará en Haskell. El contenido de este capítulo
se encuentra en el módulo \texttt{RES}
\entrada{RES}
\chapter{Correspondencia de Curry-Howard}

En este capítulo trataremos la correspondencia de Curry-Howard, también llamada isomorfismo o equivalencia de Curry-Howard. Debe su nombre a que establece una correspondencia entre las pruebas en la lógica y los tipos de datos en la programación. 

En concreto, de forma informal la correspondencia de Curry-Howard establece que:
\begin{itemize*}
\item Los tipos de datos corresponden a proposiciones.
\item Los valores corresponden a pruebas lógicas. 
\end{itemize*}

A continuación, presentemos una tabla resumen, aunque luego hablaremos de cada entrada de la tabla con más detenimiento con las correspondencias entre elementos de lógica y la programación en Haskell. 


 \begin{center}
   \begin{tabular}{| l | l |}
     \hline
      Pruebas            & Programas \\ \hline
      Fórmulas            & Tipos \\ \hline
      $P\Rightarrow Q$          & \texttt{f:: P->Q} \\ \hline
      $a\wedge b$     &  \texttt{(a,b)} \\ \hline
      $a\vee b$ & \texttt{Either a b}\\ \hline
      $T$    & \texttt{()} \\ \hline
     $F$ & \texttt{absurd} \\ \hline
   \end{tabular}
 \end{center}

\entrada{CHC}
\appendix

\chapter{Trabajando con GitHub}

En este apéndice se pretende introducir al lector en el empleo de
\href{https://github.com/}{GitHub}, sistema remoto de versiones.

\section{Crear una cuenta}

El primer paso es crear una cuenta en la página web de
\href{https://github.com/}{GitHub}, para ello \fbox{\texttt{sign up}} y se
rellena el formulario.

\section{Crear un repositorio}

Mediante \fbox{\texttt{New repository}} se crea un repositorio nuevo.  Un
repositorio es una carpeta de trabajo. En ella se guardarán todas las versiones
y modificaciones de nuestros archivos.

Necesitamos darle un nombre adecuado y seleccionar
\begin{enumerate}
\item En \fbox{\texttt{Add .gitignore}} se selecciona \texttt{Haskell}.
\item En \fbox{\texttt{add a license}} se selecciona \texttt{GNU General Public License v3.0}.
\end{enumerate}

Finalmente \fbox{\texttt{Create repository}}

\section{Conexión}

Nuestro interés está en poder trabajar de manera local y poder tanto actualizar
GitHub como nuestra carpeta local. Los pasos a seguir son
\begin{enumerate}
\item Generar una clave SSH mediante el comando
\begin{verbatim}
   ssh-keygen -t rsa -b 4096 -C "tuCorreo"
\end{verbatim}
  Indicando una contraseña. Si queremos podemos dar una localización de guardado
  de la clave pública.
\item Añadir la clave a \texttt{ssh-agent}, mediante
\begin{verbatim}
   eval "$(ssh-agent -s)"
   ssh-add ~/.ssh/id_rsa
\end{verbatim}

\item Añadir la clave a nuestra cuenta. Para ello: \texttt{Setting}
  $\rightarrow$ \texttt{SSH and GPG keys}
  $\rightarrow$ \fbox{\texttt{New SSH key}}. En esta última sección se copia el
  contenido de
\begin{verbatim}
   ~/.ssh/id_rsa.pub
\end{verbatim}
 por defecto. Podríamos haber puesto otra localización en el primer paso.
\item Se puede comprobar la conexión mediante el comando
\begin{verbatim}
   ssh -T git@github.com
\end{verbatim}
\item Se introducen tu nombre y correo
\begin{verbatim}
   git config --global user.name "Nombre"
   git config --global user.email "<tuCorreo>"
\end{verbatim}
\end{enumerate}

\section{Pull y push}

Una vez hecho los pasos anteriores, ya estamos conectados con GitHub y podemos
actualizar nuestros ficheros. Nos resta aprender a actualizarlos.

\begin{enumerate}
\item Clonar el repositorio a una carpeta local:

  Para ello se ejecuta en una terminal
\begin{verbatim}
   git clone <enlace que obtienes en el repositorio>
\end{verbatim}
  El enlace que sale en el repositorio pinchando en \fbox{\texttt{(clone or
  download)}} y, eligiendo \fbox{\texttt{(use SSH)}}.

\item Actualizar tus ficheros con la versión de GitHub:

  En emacs se ejecuta \fbox{\texttt{(Esc-x)+(magit-status)}}. Para ver una
  lista de los cambios que están \texttt{(unpulled)}, se ejecuta en
  magit \fbox{\texttt{remote update}}. Se emplea \texttt{pull},y se actualiza.
  \texttt{(Pull: origin/master)}
\item Actualizar GitHub con tus archivos locales:

  En emacs se ejecuta \texttt{(Esc-x)+(magit-status)}. Sale la lista
  de los cambios \texttt{(UnStages)}. Con la \texttt{(s)} se guarda,
  la \texttt{(c)+(c)} hace \texttt{(commit)}. Le ponemos un título y,
  finalmente \texttt{(Tab+P)+(P)} para hacer \texttt{(push)} y subirlos
  a GitHub.
\end{enumerate}

\section{Ramificaciones (``branches'')}

Uno de los puntos fuertes de Git es el uso de ramas. Para ello,
creamos una nueva rama de trabajo. En \texttt{(magit-status)}, se pulsa
\texttt{b}, y luego \texttt{(c)} \texttt{(create a new branch and checkout)}.
Checkout cambia de rama a la nueva, a la que habrá que dar un nombre.

Se trabaja con normalidad y se guardan las modificaciones con
\texttt{magit-status}. Una vez acabado el trabajo, se hace \texttt{(merge)}
con la rama principal y la nueva.

Se cambia de rama \texttt{(branch...)} y se hace \texttt{(pull)}
como acostumbramos.

Finalmente, eliminamos la rama mediante \texttt{(magit-status)}
$\rightarrow $ \texttt{(b)} $\rightarrow$ \texttt{(k)}
$\rightarrow $ \texttt{(Nombre de la nueva rama)}

%%% Local Variables:
%%% mode: latex
%%% TeX-master: "LPO_en_Haskell"
%%% End:

\chapter{Usando Doctest}\label{aped.C}

En este apéndice se introducirá el uso del paquete doctest. Fue implementado
basado en el \href{https://docs.python.org/3/library/doctest.html}{paquete doctest para Python}.

Cuando se realizan trabajos de programación de gran extensión en el que muchos programas
dependen unos de otros, el trabajo de documentación y comprobación puede volverse un caos.
Un sistema para llevar a cabo estas penosas tareas es empleando el \href{https://hackage.haskell.org/package/doctest}{paquete doctest}. Su documentación se encuentra en \cite{DoctestDoc}, así como una guía más amplia de la que aquí se expone.


El paquete \texttt{doctest} se puede encontrar en \href{http://hackage.haskell.org/package/doctest}{Hackage} y, por lo tanto, se puede instalar mediante \fbox{\texttt{cabal install doctest}}.

En cuanto a su uso, los ejemplos deben ser estructurados tal y como se ha realizado en todo el trabajo. Por ejemplo, en resolución lo hicimos de la siguiente manera:

\entrada{doctest}
%%%%%%%%%%%%%%%%%%%%%%%%%%%%%%%%%%%%%%%%%%%%%%%%%%%%%%%%%%%%%%%%%%%%%%%%%%%%%%% 
%%  Bibliografía                                                            %%
%%%%%%%%%%%%%%%%%%%%%%%%%%%%%%%%%%%%%%%%%%%%%%%%%%%%%%%%%%%%%%%%%%%%%%%%%%%%%%%

\addcontentsline{toc}{chapter}{Bibliografía}
\bibliographystyle{abbrv}
\bibliography{LPO_en_Haskell}

%%%%%%%%%%%%%%%%%%%%%%%%%%%%%%%%%%%%%%%%%%%%%%%%%%%%%%%%%%%%%%%%%%%%%%%%%%%%%%%
%%  Índice                                                                  %%
%%%%%%%%%%%%%%%%%%%%%%%%%%%%%%%%%%%%%%%%%%%%%%%%%%%%%%%%%%%%%%%%%%%%%%%%%%%%%%%

\addcontentsline{toc}{chapter}{Indice de definiciones}

\printindex

%%%%%%%%%%%%%%%%%%%%%%%%%%%%%%%%%%%%%%%%%%%%%%%%%%%%%%%%%%%%%%%%%%%%%%%%%%%%%%%
%% § Pendientes                                                              %%
%%%%%%%%%%%%%%%%%%%%%%%%%%%%%%%%%%%%%%%%%%%%%%%%%%%%%%%%%%%%%%%%%%%%%%%%%%%%%%%

\todototoc
\listoftodos

\end{document}



%%% Local Variables:
%%% mode: latex
%%% TeX-master: t
%%% End:
