\chapter{Sintaxis y semántica de la lógica de primer orden}

El \textbf{lenguaje de la lógica de primer orden} está compuesto por

\begin{enumerate}
\item Variables proposicionales $p,q,r,\dots$
\item Conectivas lógicas:
  \begin{center}
   \begin{tabular}{| l | l |}
     \hline
      $\neg$            & \text{Negación} \\ \hline
      $\vee$            & \text{Disyunción} \\ \hline
      $\wedge$          & \text{Conjunción} \\ \hline
      $\rightarrow$     & \text{Condicional} \\ \hline
      $\leftrightarrow$ & \text{Bicondicional}\\
     \hline
   \end{tabular}
 \end{center}
\item Símbolos auxiliares $"(",")"$
\item Cuantificadores: $\forall$ (Universal) y $\exists$ (Existencial)
\item Símbolo de igualdad: $=$
\item Constantes: $a,b,\dots,a_1,a_2,\dots$
\item Símbolos de relación: $P,Q,R,\dots $
\item Símbolos de función: $f,g,h\dots $
\end{enumerate}

\entrada{Modelo}.
