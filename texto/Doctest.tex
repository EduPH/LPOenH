\chapter{Usando Doctest}\label{aped.C}

En este apéndice se introducirá el uso del paquete doctest. Fue implementado
basado en el \href{https://docs.python.org/3/library/doctest.html}{paquete doctest para Python}.

Cuando se realizan trabajos de programación de gran extensión en el que muchos programas
dependen unos de otros, el trabajo de documentación y comprobación puede volverse un caos.
Un sistema para llevar a cabo estas penosas tareas es empleando el \href{https://hackage.haskell.org/package/doctest}{paquete doctest}. Su documentación se encuentra en \cite{DoctestDoc}, así como una guía más amplia de la que aquí se expone.


El paquete \texttt{doctest} se puede encontrar en \href{http://hackage.haskell.org/package/doctest}{Hackage} y, por lo tanto, se puede instalar mediante \fbox{\texttt{cabal install doctest}}.

En cuanto a su uso, los ejemplos deben ser estructurados tal y como se ha realizado en todo el trabajo. Por ejemplo, en resolución lo hicimos de la siguiente manera:

\entrada{doctest}