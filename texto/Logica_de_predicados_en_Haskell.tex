\chapter{Lógica de predicados en Haskell}

En este capítulo se estudiará la lógica de predicados en Haskell, para ello
necesitamos de un preámbulo de definiciones necesarias, propias de la lógica.

\section{Conceptos previos}

Primero, debemos conocer la lógica proposicional. El alfabeto de la lógica
proposicional está compuesto por

\begin{enumerate}
\item Variables proposicionales
\item Conectivas lógicas:
  \begin{center}
   \begin{tabular}{| l | l |}
     \hline
      $\neg$   & \text{Negación} \\ \hline
      $\vee$   & \text{Disyunción} \\ \hline
      $\wedge$ & \text{Conjunción} \\ \hline
      $\rightarrow$ & \text{Condicional} \\ \hline
      $\leftrightarrow$ & \text{Bicondicional}\\
     \hline
   \end{tabular}
 \end{center}
\end{enumerate}

\begin{Def}
  Se dice que \texttt{F} es una \textbf{fórmula} si satisface la siguiente definición
  inductiva
  \begin{enumerate}
  \item Las variables proposicionales son fórmulas atómicas.
  \item Si $F$ y $G$ son fórmulas, entonces $\neg F$, $(F \wedge G)$,
    $(F \vee G)$, $(F \rightarrow G)$ y $(F \leftrightarrow G)$ son fórmulas.
\end{enumerate}
\end{Def}

\begin{Def}
  Un \textbf{predicado} es una oración narrativa que puede ser verdadera o falsa.
\end{Def}

\begin{Def}
  Una fórmula está en forma \textbf{normal conjuntiva} si es una conjunción de
  disyunciones de literales.
  $$(p_1\vee \dots \vee p_n)\wedge \dots \wedge (q_1\vee \dots \vee q_m)$$
\end{Def}

\begin{Def}
  Una fórmula está en forma \textbf{normal disyuntiva} si es una disyunción de
  conjunciones de literales.
  $$(p_1 \wedge \dots \wedge p_n)\vee \dots \vee (q_1\wedge \dots \wedge q_m)$$
\end{Def}

Nosotros trabajaremos con la lógica de primer orden. Para ello, debemos añadir
a los conceptos ya introducidos de la lógica proposicional los siguientes
elementos
\begin{enumerate}
\item Cuantificadores: $\forall$ (Universal) y $\exists$ (Existencial)
\item Símbolo de igualdad: $=$
\item Constantes: $a,b,\dots,a_1,a_2,\dots$
\item Predicados
\item Funciones
\end{enumerate}
\entrada{Modelo}.

%%% Local Variables:
%%% mode: latex
%%% TeX-master: "LPO_en_Haskell"
%%% End:
