\chapter*{Introducción}
\addcontentsline{toc}{chapter}{Introducción}

El objetivo del trabajo es la implementación de los algoritmos de la lógica de
primer orden en Haskell. Consta de dos partes:

\begin{itemize}
\item La primera parte consiste en la implementación en Haskell de la teoría impartida en la asignatura \href{https://www.cs.us.es/~jalonso/cursos/lmf-15}
       {Lógica matemática y fundamentos}\
  \footnote{\url{{https://www.cs.us.es/~jalonso/cursos/lmf-15}}}
  (\cite{Alonso-15a}). Para ello, se lleva a cabo la adaptación de los programas del libro de
  J. van Eijck 
  \href{http://citeseerx.ist.psu.edu/viewdoc/download?doi=10.1.1.467.1441&rep=rep1&type=pdf}
       {Computational semantics and type theory}\
  \footnote{\url{http://citeseerx.ist.psu.edu/viewdoc/download?doi=10.1.1.467.1441&rep=rep1&type=pdf}}
  (\cite{Eijck-03}) y su correspondiente teoría. 

\item En la segunda parte se comentan aquellos sistemas empleados en la elaboración del
  trabajo.
 
\end{itemize}



%%% Local Variables: 
%%% mode: latex
%%% TeX-master: "LPO_en_Haskell"
%%% End: 
