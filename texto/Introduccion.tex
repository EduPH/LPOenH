\chapter*{Introducción}
\addcontentsline{toc}{chapter}{Introducción}

La lógica de primer orden o lógica de predicados nace como una extensión de la
lógica proposicional ante algunas carencias que ésta presenta. La lógica
proposicional tiene como objetivo modelizar el razonamiento y nos aporta una
manera de formalizar las demostraciones. El problema surge cuando aparecen
propiedades universales o nos cuestionamos la existencia de elementos con una
cierta propiedad, es decir, aparecen razonamientos del tipo
\begin{center}
\begin{tabular}{l}
  Todos los sevillanos van a la feria. \\
  Yo soy sevillano.\\
  Por lo tanto, voy a la feria.
\end{tabular}
\end{center}
A pesar de ser yo una clara prueba de que la proposición no es cierta, sí que
se infieren las carencias de la lógica proposicional, la ausencia de
cuantificadores.

\vspace{4mm}

En este trabajo se pretende realizar una implementación en Haskell de la teoría
impartida en la asignatura
\href{https://www.cs.us.es/~jalonso/cursos/lmf-15}{Lógica matemática y
  fundamentos}\ \footnote{\url{{https://www.cs.us.es/~jalonso/cursos/lmf-15}}}
(\cite{Alonso-15a}) del grado en matemáticas.  Para ello, se lleva a cabo la
adaptación de los programas del libro de J. van Eijck
\href{http://citeseerx.ist.psu.edu/viewdoc/download?doi=10.1.1.467.1441&rep=rep1&type=pdf}
{Computational semantics and type theory}\
\footnote{\url{http://citeseerx.ist.psu.edu/viewdoc/download?doi=10.1.1.467.1441&rep=rep1&type=pdf}}
(\cite{Eijck-03}) y su correspondiente teoría.

\vspace{3mm}

Se ha comenzado con una \textbf{introducción a Haskell}(\ref{sec:progfunHas}) a
través de la definición de funciones y numerosos ejemplos. Esto lleva a
introducir las funciones básicas, la teoría de tipos y la teoría de
generadores. De esta manera, se establece la base de conocimientos propios de
la programación que van más allá de la lógica y que son necesarios para la
implementación.

A continuación, en el capítulo sobre \textbf{sintaxis y semántica de la lógica
de primer orden} (\ref{sec:sinsemlpo}), ilustramos la utilidad de la lógica
para dar una representación del conocimiento proponiendo un modelo de
clasificación botánica. Se construyen los tipos de datos abstractos que sirven
de vehículo en la representación de fórmulas y que son base de las
implementaciones en los restantes capítulos. También se definen la evaluación
de fórmulas por una interpretación y los términos funcionales, extendiendo el
lenguaje y dando, posteriormente, la construcción de generadores de fórmulas
para, en caso necesario, poder hacer comprobaciones con QuickCheck.

Ya asentadas las bases de la lógica y teniendo un suelo en el que sostenernos,
se trabaja con los \textbf{tableros semánticos} (\ref{sec:tableros}). Dicho
capítulo tiene como objetivo implementar el método de tableros, un algoritmo
para la prueba de teoremas y búsqueda de modelos. Para la definición del
algoritmo se requiere de distintas definiciones previas:

\begin{itemize}
\item Sustitución: Nos permite llevar a cabo sustituciones de variables libres
  por términos.
\item Unificación: Se basa en la posibilidad de sustituir términos en varias
  fórmulas para unificarlas; es decir, que después de la sustitución sean
  sintácticamente iguales. Para ello determinamos tanto su existencia como
  generamos la lista de unificadores.
\item Forma de Skolem: Se trata de una forma de representación de una fórmula
  dada que se obtiene a partir de una serie de reglas, a través de las cuales
  se consiguen representaciones equiconsistentes unas de otras de la misma
  fórmula. En el proceso de definición de la forma de Skolem se definen las
  formas normales, forma rectificada, forma normal prenexa y forma normal
  prenexa conjuntiva. Éstas además son necesarias para la definición en un
  capítulo posterior de la forma clausal.
\end{itemize}

Posteriormente hablamos sobre un método constructivo para generar
interpretaciones de fórmulas o conjuntos de fórmulas, es lo que llamamos
\textbf{modelos de Herbrand} (\ref{sec:herbrand}), que debe su nombre al
matemático francés Jacques Herbrand. Además, es un método para determinar la
consistencia de un conjunto de cláusulas, pues tener un modelo de Herbrand es
equivalente a consistencia.

Luego implementamos la \textbf{resolución} (\ref{sec:resolucion}), que
establece una regla para obtener deducciones lógicas a partir de
cláusulas. Para ello, definimos la forma clausal y su interpretación con el
objetivo de implementar las resolventes binarias, tanto para la lógica
proposicional como la lógica de primer orden, siendo éstas un método para
obtener deducciones de dos cláusulas. En ese capítulo se construye la base de
la resolución pero no se implementan algoritmos de demostración vía resolución,
los cuales podrían establecerse basados en el código propuesto.
  
Todos estos capítulos antes mencionados suponen la implementación de una rama
matemática en un lenguaje de programación, y eso nos lleva al último de ellos
que trata sobre la \textbf{correspondencia de Curry-Howard}
(\ref{sec:curry}). Así, vemos que hemos establecido una relación entre
matemáticas y programación, que se ve reflejada en el isomorfismo de
Curry-Howard que establece que las proposiciones son tipos y las demostraciones
funciones, siendo matemáticas y programación dos caras de una misma
moneda. Además, se propone la analogía entre distintos ejemplos que permiten
convencerse de la veracidad del isomorfismo.

\vspace{3mm}

Finalmente, cabe comentar que la elaboración de este trabajo ha supuesto el
aprendizaje y el empleo de distintos métodos y recursos.

La escritura del trabajo se ha llevado a cabo en Haskell literario. La
programación literaria es un estilo de programación propuesto por Donald Knuth,
un experto en ciencias de la computación y autor de TeX. En nuestro caso,
supone la escritura simultánea de código y texto vía LaTeX y Haskell,
resultando un documento en el que el código es ejecutable y en su totalidad
reproducible según las características de LaTeX.

Para la cooperación entre el tutor y el alumno se ha realizado el trabajo en la
plataforma GitHub, en el repositorio
\href{https://github.com/EduPH/LPOenH}{LPOenH}. Esto ha llevado a un
aprendizaje del uso de \href{https://git-scm.com/}{\texttt{git}}, un sistema de
control de versiones para el manejo de proyectos de manera rápida y
eficiente. Además, se ha realizado una pequeña guía para su uso en emacs en el
apéndice \ref{aped.A}.

También cabe comentar que todos los ejemplos y definiciones del trabajo se han
comprobado empleando la librería \texttt{doctest}, habiendo realizado además
una pequeña guía en el apéndice \ref{aped.B}.

%%% Local Variables: 
%%% mode: latex
%%% TeX-master: "LPO_en_Haskell"
%%% End: 
