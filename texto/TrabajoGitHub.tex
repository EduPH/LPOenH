\chapter{Apéndice: GitHub}

En este apéndice se pretende introducir al lector en el empleo de
\href{https://github.com/}{GitHub}, sistema remoto de versiones.

\section{Crear una cuenta}

El primer paso es crear una cuenta en la página web de
\href{https://github.com/}{GitHub}, para ello \fbox{\texttt{sign up}} y se
rellena el formulario.

\section{Crear un repositorio}

Mediante \fbox{\texttt{New repository}} se crea un repositorio nuevo.  Un
repositorio es una carpeta de trabajo. En ella se guardarán todas las versiones
y modificaciones de nuestros archivos.

Necesitamos darle un nombre adecuado y seleccionar
\begin{enumerate}
\item En \fbox{\texttt{Add .gitignore}} se selecciona \texttt{Haskell}.
\item En \fbox{\texttt{add a license}} se selecciona \texttt{GNU General Public License v3.0}.
\end{enumerate}

Finalmente \fbox{\texttt{Create repository}}

\section{Conexión}

Nuestro interés está en poder trabajar de manera local y poder tanto actualizar
GitHub como nuestra carpeta local. Los pasos a seguir son
\begin{enumerate}
\item Generar una clave SSH mediante el comando
\begin{verbatim}
   ssh-keygen -t rsa -b 4096 -C "tuCorreo"
\end{verbatim}
  Indicando una contraseña. Si queremos podemos dar una localización de guardado
  de la clave pública.
\item Añadir la clave a \texttt{ssh-agent}, mediante
\begin{verbatim}
   eval "$(ssh-agent -s)"
   ssh-add ~/.ssh/id_rsa
\end{verbatim}

\item Añadir la clave a nuestra cuenta. Para ello: \texttt{Setting}
  $\rightarrow$ \texttt{SSH and GPG keys}
  $\rightarrow$ \fbox{\texttt{New SSH key}}. En esta última sección se copia el
  contenido de
\begin{verbatim}
   ~/.ssh/id_rsa.pub
\end{verbatim}
 por defecto. Podríamos haber puesto otra localización en el primer paso.
\item Se puede comprobar la conexión mediante el comando
\begin{verbatim}
   ssh -T git@github.com
\end{verbatim}
\item Se introducen tu nombre y correo
\begin{verbatim}
   git config --global user.name "Nombre"
   git config --global user.email "<tuCorreo>"
\end{verbatim}
\end{enumerate}

\section{Pull y push}

Una vez hecho los pasos anteriores, ya estamos conectados con GitHub y podemos
actualizar nuestros ficheros. Nos resta aprender a actualizarlos.

\begin{enumerate}
\item Clonar el repositorio a una carpeta local:

  Para ello se ejecuta en una terminal
\begin{verbatim}
   git clone <enlace que obtienes en el repositorio>
\end{verbatim}
  El enlace que sale en el repositorio pinchando en \fbox{\texttt{(clone or
  download)}} y, eligiendo \fbox{\texttt{(use SSH)}}.

\item Actualizar tus ficheros con la versión de GitHub:

  En emacs se ejecuta \fbox{\texttt{(Esc-x)+(magit-status)}}. Para ver una
  lista de los cambios que están \texttt{(unpulled)}, se ejecuta en
  magit \fbox{\texttt{remote update}}. Se emplea \texttt{pull},y se actualiza.
  \texttt{(Pull: origin/master)}
\item Actualizar GitHub con tus archivos locales:

  En emacs se ejecuta \texttt{(Esc-x)+(magit-status)}. Sale la lista
  de los cambios \texttt{(UnStages)}. Con la \texttt{(s)} se guarda,
  la \texttt{(c)+(c)} hace \texttt{(commit)}. Le ponemos un título y,
  finalmente \texttt{(Tab+P)+(P)} para hacer \texttt{(push)} y subirlos
  a GitHub.
\end{enumerate}

\section{Ramificaciones (``branches'')}

Uno de los puntos fuertes de Git es el uso de ramas. Para ello,
creamos una nueva rama de trabajo. En \texttt{(magit-status)}, se pulsa
\texttt{b}, y luego \texttt{(c)} \texttt{(create a new branch and checkout)}.
Checkout cambia de rama a la nueva, a la que habrá que dar un nombre.

Se trabaja con normalidad y se guardan las modificaciones con
\texttt{magit-status}. Una vez acabado el trabajo, se hace \texttt{(merge)}
con la rama principal y la nueva.

Se cambia de rama \texttt{(branch...)} y se hace \texttt{(pull)}
como acostumbramos.

Finalmente, eliminamos la rama mediante \texttt{(magit-status)}
$\rightarrow $ \texttt{(b)} $\rightarrow$ \texttt{(k)}
$\rightarrow $ \texttt{(Nombre de la nueva rama)}

%%% Local Variables:
%%% mode: latex
%%% TeX-master: "LPO_en_Haskell"
%%% End:
