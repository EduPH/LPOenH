\chapter{Correspondencia de Curry-Howard}

En este capítulo trataremos la correspondencia de Curry-Howard, también llamada isomorfismo o equivalencia de Curry-Howard. Debe su nombre a que establece una correspondencia entre las pruebas en la lógica y los tipos de datos en la programación. 

En concreto, de forma informal la correspondencia de Curry-Howard establece que:
\begin{itemize*}
\item Los tipos de datos corresponden a proposiciones.
\item Los valores corresponden a pruebas lógicas. 
\end{itemize*}

A continuación, presentemos una tabla resumen, aunque luego hablaremos de cada entrada de la tabla con más detenimiento con las correspondencias entre elementos de lógica y la programación en Haskell. 


 \begin{center}
   \begin{tabular}{| l | l |}
     \hline
      Pruebas            & Programas \\ \hline
      Fórmulas            & Tipos \\ \hline
      $P\Rightarrow Q$          & \texttt{f:: P->Q} \\ \hline
      $a\wedge b$     &  \texttt{(a,b)} \\ \hline
      $a\vee b$ & \texttt{Either a b}\\ \hline
      $T$    & \texttt{()} \\ \hline
     $F$ & \texttt{absurd} \\ \hline
   \end{tabular}
 \end{center}

\entrada{CHC}