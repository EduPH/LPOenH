\documentclass[final]{beamer}
\usepackage[utf8x]{inputenc}       % Acentos de UTF8
\usepackage[spanish]{babel}        % Castellanización.
\usepackage[T1]{fontenc}           % Codificación T1 con European Computer
                                   % Modern.  
\usepackage{graphicx}
\usepackage{fancyvrb}              % Verbatim extendido
\usepackage{makeidx}               % Índice
\usepackage{amsmath}               % AMS LaTeX
\usepackage{amsthm} 
\usepackage{latexsym}
\usepackage{mathpazo}              % Fuentes semejante a palatino
\usepackage[scaled=.90]{helvet}
\usepackage{cmtt}
\input definiciones

\usepackage[scale=1.24]{beamerposter} % Use the beamerposter package for laying out the poster

\usetheme{confposter} % Use the confposter theme supplied with this template

\setbeamercolor{block title}{fg=ngreen,bg=white} % Colors of the block titles
\setbeamercolor{block body}{fg=black,bg=white} % Colors of the body of blocks
\setbeamercolor{block alerted title}{fg=white,bg=dblue!70} % Colors of the highlighted block titles
\setbeamercolor{block alerted body}{fg=black,bg=dblue!10} % Colors of the body of highlighted blocks
% Many more colors are available for use in beamerthemeconfposter.sty
%-----------------------------------------------------------
% Define the column widths and overall poster size
% To set effective sepwid, onecolwid and twocolwid values, first choose how many columns you want and how much separation you want between columns
% In this template, the separation width chosen is 0.024 of the paper width and a 4-column layout
% onecolwid should therefore be (1-(# of columns+1)*sepwid)/# of columns e.g. (1-(4+1)*0.024)/4 = 0.22
% Set twocolwid to be (2*onecolwid)+sepwid = 0.464
% Set threecolwid to be (3*onecolwid)+2*sepwid = 0.708

\newlength{\sepwid}
\newlength{\onecolwid}
\newlength{\twocolwid}
\newlength{\threecolwid}
\setlength{\paperwidth}{48in} % A0 width: 46.8in
\setlength{\paperheight}{36in} % A0 height: 33.1in
\setlength{\sepwid}{0.024\paperwidth} % Separation width (white space) between columns
\setlength{\onecolwid}{0.22\paperwidth} % Width of one column
\setlength{\twocolwid}{0.464\paperwidth} % Width of two columns
\setlength{\threecolwid}{0.708\paperwidth} % Width of three columns
\setlength{\topmargin}{-0.5in} % Reduce the top margin size
%-----------------------------------------------------------

\usepackage{graphicx}  % Required for including images

\usepackage{booktabs} % Top and bottom rules for tables
%----------------------------------------------------------------------------------------
%	Título
%----------------------------------------------------------------------------------------

\title{Lógica de primer orden en Haskell} % Poster title

\author{Eduardo Paluzo Hidalgo} % Author(s)

\institute{Universidad de Sevilla} % Institution(s)

%----------------------------------------------------------------------------------------

\begin{document}

\addtobeamertemplate{block end}{}{\vspace*{2ex}} % White space under blocks
\addtobeamertemplate{block alerted end}{}{\vspace*{2ex}} % White space under highlighted (alert) blocks

\setlength{\belowcaptionskip}{2ex} % White space under figures
\setlength\belowdisplayshortskip{2ex} % White space under equations

\begin{frame}[t,fragile] % The whole poster is enclosed in one beamer frame

\begin{columns}[t] % The whole poster consists of three major columns, the second of which is split into two columns twice - the [t] option aligns each column's content to the top

\begin{column}{\sepwid}\end{column} % Empty spacer column

\begin{column}{\onecolwid} % The first column

%----------------------------------------------------------------------------------------
%	Contenido
%----------------------------------------------------------------------------------------

\begin{alertblock}{Contenido}

  En este trabajo se han implementado las bases de la lógica de primer orden, así como
  una serie de algoritmos. En concreto:
  
\begin{itemize}
\item Sintáxis y semántica de la lógica de primer orden.
\item Método de los tableros semánticos.
\item Modelos de Herbrand.
\item Resolución en lógica de primer orden.
\item Correspondencia de Curry-Howard.
\end{itemize}

\end{alertblock}

%----------------------------------------------------------------------------------------
%	Introducción
%----------------------------------------------------------------------------------------

\begin{block}{Introducción}

  La lógica de primer orden surge como una extensión de la lógica proposicional
  ante carencias que ésta presenta. Aquí se ha realizado una implementación
  de la lógica de primer orden, de su lenguaje y de ciertos algoritmos.
  Además, se ha establecido una relación entre las matemáticas y la
  programación motivada por la realización de esta implementación.
  

\end{block}
%-----------------------------------------------
%  Sintáxis y semántica
%------------------------------------------------

\begin{block}{Sintáxis y semántica}

  La lógica de primer orden presenta cuantificadores, siendo éste el principal elemento
  que la diferencia con la proposicional. Por ello, lo primero que se realizó es
  asentar su base y representación en Haskell.

\begin{code}
data Form = Atom Nombre [Termino]
          | Ig Termino Termino
          | Neg Form
          | Impl Form Form
          | Equiv Form Form
          | Conj [Form]
          | Disy [Form]
          | PTodo Variable Form
          | Ex Variable Form
  deriving (Eq,Ord,Generic)  
\end{code}

Definiendo su representación en la clase \texttt{Show}, se obtienen fórmulas de este tipo:

\begin{sesion}
ghci> PTodo x (Ex y (Impl (Atom "R"[tx]) (Atom "P" [ty])))
∀x ∃y (R[x]⟹P[y])
\end{sesion}

Posteriormente, se definieron los términos funcionales y la
evaluación de fórmulas.

\end{block}
\begin{block}{Evaluación de fórmulas}
  Dada una estructura $\mathcal{I} = (U,I)$ de $L$ y una asignación
  $A$ sobre $\mathcal{I}$, se define la \textbf{función evaluación de fórmulas}
  $\mathcal{I}_A: Form(L) \rightarrow Bool$ por
  \begin{itemize*}
  \item Si $F$ es $t_1=t_2$,  
    $\mathcal{I}_A(F) = H_=(\mathcal{I}_A(t_1),I_A(t_2))$
  \item Si $F$ es $P(t_1,\dots ,t_n)$,  
    $\mathcal{I}_A(F) = H_{I(P)}(\mathcal{I}_A(t_1), \dots ,\mathcal{I}_A(t_n))$
  \item Si $F$ es $\neg G$, 
    $\mathcal{I}_A(F) = H_{\neg}(\mathcal{I}_A(G))$
  \item Si $F$ es $G*H$, 
    $\mathcal{I}_A(F) = H_*(\mathcal{I}_A(G),\mathcal{I}_A(H))$
  \item Si $F$ es $\forall x G$,
    \begin{equation*}
      \mathcal{I}_A(F) = \left\{
      \begin{array}{ll} 
        1, \text{ si para todo } u \in U \text{ se tiene } \mathcal{I}_{A [ x / u ]} = 1 \\
        0, \text{ en caso contario}.
      \end{array} \right.
    \end{equation*}
  \item Si $F$ es $\exists x G$,
     \begin{equation*}
      \mathcal{I}_A(F) = \left\{
      \begin{array}{ll}
        1, \text{ si existe algún } u \in U \text{ tal que } 
           \mathcal{I}_{A [x / u ]} = 1 \\
        0, \text{ en caso contario}.
      \end{array} \right.
    \end{equation*}
  \end{itemize*}
\end{block}


 
------------------------------------------------------------------------

\end{column} % End of the first column

\begin{column}{\sepwid}\end{column} % Empty spacer column

\begin{column}{\twocolwid} % Begin a column which is two columns wide (column 2)

\begin{columns}[t,totalwidth=\twocolwid] % Split up the two columns wide column

\begin{column}{\onecolwid}\vspace{-.6in} % The first column within column 2 (column 2.1)

  \begin{block}{Implementación}
    \begin{code}
valorF :: Eq a => Universo a -> Interpretacion a
             -> Asignacion a -> Form -> Bool
valorF u (iR,iF) a (Atom r ts) =
  iR r (map (valorT iF a) ts)
valorF u (_,iF) a (Ig t1 t2) = 
  valorT iF a t1 == valorT iF a t2
valorF u i a (Neg g) =
  not (valorF u i a g)
valorF u i a (Impl f1 f2) =
  valorF u i a f1 <= valorF u i a f2
valorF u i a (Equiv f1 f2) =
  valorF u i a f1 == valorF u i a f2
valorF u i a (Conj fs) =
  all (valorF u i a) fs
valorF u i a (Disy fs) =
  any (valorF u i a) fs
valorF u i a (PTodo v g) = 
  and [valorF u i (sustituye a v d) g | d <- u]
valorF u i a (Ex v g)  =
  or  [valorF u i (sustituye a v d) g | d <- u]
    \end{code}
  
  \end{block}

  \begin{block}{Generadores}
    Se han implementado generadores de los tipos construidos para la lógica de primer orden.
    Pudiendo, en caso necesario, realizar comprobaciones con \texttt{QuickCheck} sobre
    propiedades.
  \end{block}

%----------------------------------------------------------------------------------------

\end{column} % End of column 2.1

\begin{column}{\onecolwid}\vspace{-.6in} % The second column within column 2 (column 2.2)

% ---------------------------------------------------------------------------------------
% TABLEROS SEMÁNTICOS
%---------------------------------------------------------------------------------------

  \begin{block}{Tableros semánticos}
    El método de los tableros semánticos se emplea para la demostración de teoremas en
    lógica de predicados. Para su implementación, se requirió de la implementación de:

    \begin{itemize}
    \item \textbf{Sustitución:} Para cambiar variables por términos en una fórmula.
    \item \textbf{Unificación:} Se trata de sustituciones que mantienen equivalencias
      en las fórmulas y que permiten, en caso de existencia, renombrar términos con el
      mismo nombre.
    \item \textbf{Forma de Skolem:} Empleando equivalencias entre fórmulas, se obtiene
      una representación ($\forall x_1 \dots \forall x_n G$) donde $G$ no tiene
      cuantificadores. 
    \end{itemize}
    Una vez realizadas todas estas implementaciones, se puede definir el método de tableros,
    consiguiendo así un demostrador. El algoritmo es el siguiente:

    \begin{itemize*}
\item El árbol cuyo único nodo tiene como etiqueta $S$ es
  un tablero de $S$.
\item Sea $\mathcal{T}$ un tablero de $S$ y $S_1$ la etiqueta de
  una hoja  de $\mathcal{T}$.
  \begin{enumerate}
  \item Si $S_1$ contiene una fórmula y su negación, entonces el árbol
    obtenido añadiendo como hijo de $S_1$ el nodo etiquetado con
    $\left\{ \perp \right\}$ es un tablero de $S$.
  \item Si $S_1$ contiene una doble negación $\neg \neg F$, entonces
    el árbol obtenido añadiendo como hijo de $S_1$ el nodo etiquetado
    con $(S_1 \backslash \left\{\neg \neg F \right\})\cup \left\{ F \right\}$
    es un tablero de $S$.
  \item Si $S_1$ contiene una fórmula alfa $F$ de componentes $F_1$ y $F_2$,
    entonces el árbol obtenido añadiendo como hijo de $S_1$ el nodo etiquetado
    con $(S_1 \backslash \left\{ F \right\})\cup \left\{ F_1,F_2 \right\}$
    es un tablero de $S$.
  \item Si $S_1$ contiene una fórmula beta de $F$ de componentes $F_1$ y $F_2$,
    entonces el árbol obtenido añadiendo como hijos de $S_1$ los nodos
    etiquetados con $(S_1 \backslash \left\{ F \right\}) \cup \left\{ F_1 \right\}$ y
    $(S_1 \backslash \left\{ F \right\})\cup \left\{ F_2 \right\}$ es un tablero
    de $S$.
  \end{enumerate}
\end{itemize*}
    
    
  \end{block}


\end{column} % End of column 2.2

\end{columns} % End of the split of column 2 - any content after this will now take up 2 columns width


%----------------------------------------------------------------------------------------
%	Centro
%----------------------------------------------------------------------------------------
\begin{alertblock}{Implementación}
  \begin{code}
ramificacion :: Nodo -> Tablero
ramificacion (Nd i pos neg []) = [Nd i pos neg []]
ramificacion (Nd i pos neg (f:fs)) 
  | atomo f    = if literalATer f `elem` neg
                 then []
                 else [Nd i ( literalATer f : pos) neg fs]
  | negAtomo f = if literalATer f `elem` pos
                 then []
                 else [Nd i pos (literalATer f : neg) fs]
  | dobleNeg f = [Nd i pos neg (componentes f ++ fs)]
  | alfa f     = [Nd i pos neg (componentes f ++ fs)]
  | beta f     = [Nd (i++[n]) pos neg (f':fs)
                 | (f',n) <- zip (componentes f) [0..]]
  | gamma f    = [Nd i pos neg (f':(fs++[f]))]
  where
    (xs,g) = descomponer f
    b      = [(Variable nombre j, Var (Variable nombre i)) 
             | (Variable nombre j) <- xs]
    f'     = sustitucionForm b g
  \end{code}
\end{alertblock} 
%----------------------------------------------------------------------------------------

  


\begin{columns}[t,totalwidth=\twocolwid] % Split up the two columns wide column again

\begin{column}{\onecolwid} % The first column within column 2 (column 2.1)



%----------------------------------------------------------------------------------------

\end{column} % End of column 2.1


\begin{column}{\onecolwid} % The second column within column 2 (column 2.2)

%----------------------------------------------------------------------------------------

\end{column} % End of column 2.2

\end{columns} % End of the split of column 2

\end{column} % End of the second column

\begin{column}{\sepwid}\end{column} % Empty spacer column

\begin{column}{\onecolwid} % The third column

% -------------------------------------------------------------------------------------
%  HERBRAND
% -----------------------------------------------------------------------------
  
\begin{block}{Modelos de Herbrand}
El \textbf{universo de Herbrand} de $L$ es el conjunto de términos básicos de
 $F$. Se reprenta por $\mathcal{UH}(L)$.

 Representado en Haskell por:
  
\begin{code}
universoHerbrand :: Int -> Signatura -> [Termino]
universoHerbrand 0 (cs,_,_) 
  | null cs   = [a]
  | otherwise = [Ter c [] | c <- cs]
universoHerbrand n s@(_,fs,_) =
  u `union`
  [Ter f ts | (f,k) <- fs
            , ts <- variacionesR k u]
  where u = universoHerbrand (n-1) s 
\end{code}


La \textbf{base de Herbrand} $\mathcal{BH}(L)$ de un lenguaje $L$ es el
conjunto de átomos básicos de $L$.


\begin{code}
baseHerbrand :: Int -> Signatura -> BaseH
baseHerbrand n s@(_,_,rs) =
  [Atom r ts | (r,k) <- rs
             , ts <- variacionesR k u]
  where u = universoHerbrand n s
\end{code}

   Una \textbf{interpretación de Herbrand} es una interpretación
  $\mathcal{I} = (\mathcal{U},I)$ tal que
  \begin{itemize}
  \item $\mathcal{U}$ es el universo de Herbrand de $F$.
  \item $I(c) = c$, para constante $c$ de $F$.
  \item $I(f) = f$, para cada símbolo funcional de $F$.
  \end{itemize}
Siendo su implementación:  
  \begin{code}
    interpretacionH :: AtomosH -> InterpretacionH
interpretacionH fs = (iR,iF)
  where iF f ts   = Ter  f ts
        iR r ts   = Atom r ts `elem` fs
  \end{code}
Un \textbf{modelo de Herbrand} de una fórmula $F$ es una interpretación de
Herbrand de $F$ que es modelo de $F$.
  
  Cuya implementación es:

  \begin{code}
modelosH :: Int -> [Form] -> [AtomosH]
modelosH n fs = [gs | gs <- subsequences bH
 , and [valorH uH (interpretacionH gs) f | f <- fs]]
  where uH = universoHerbrandForms n fs
        bH = baseHerbrandForms n fs
  \end{code}


\end{block}

\begin{Prop}
  Sea $S$ un conjunto de fórmulas básicas. Son equivalentes:
  \begin{enumerate}
    \item $S$ es consistente.
    \item $S$ tiene un modelo de Herbrand.
  \end{enumerate}
\end{Prop}



%----------------------------------------------------------------------------------------

\end{column} % End of the third column

\end{columns} % End of all the columns in the poster

\end{frame}

% End of the enclosing frame

% SEGUNDA DIAPOSITIVA

\begin{frame}[t,fragile] % The whole poster is enclosed in one beamer frame

\begin{columns}[t] % The whole poster consists of three major columns, the second of which is split into two columns twice - the [t] option aligns each column's content to the top

\begin{column}{\sepwid}\end{column} % Empty spacer column

\begin{column}{\onecolwid} % The first column

%----------------------------------------------------------------------------------------
%	Contenido
%----------------------------------------------------------------------------------------

\begin{block}{Forma clausal}
 Una \textbf{forma clausal} de una fórmula $F$ es un conjunto de cláusulas
 equivalente a $F$.

 Siendo el algoritmo para su cálculo:
 \begin{enumerate}
\item Sea $F_1 = \exists y_1 \dots \exists y_n F$, donde $y_i$ con
  $i=1,\dots ,n$ son las variables libres de F.
\item Sea $F_2$ una forma normal prenexa conjuntiva rectificada de $F_1$.
\item Sea $F_3= \texttt{ Skolem }(F_2)$, que tiene la forma
  $$\forall x_1 \dots \forall x_p [(L_1\vee \dots \vee L_n)
  \wedge \dots \wedge (M_1\vee \dots \vee M_m)]$$
\end{enumerate}

Y nuestra implementación:

\begin{code}
formaClausal :: Form -> Clausulas
formaClausal  = form3CAC . skolem .formaNPConjuntiva
\end{code}

Por ejemplo:

\begin{sesion}
-- | Ejemplo
-- >>> let c1 = formaClausal (Impl p q)
-- >>> let c2 = formaClausal (Impl q r)
-- >>> c1 ++! c2
-- \{\{¬p,q\},\{¬q,r\}\}
\end{sesion}
\end{block}

 
------------------------------------------------------------------------


\begin{block}{Resolución en lógica de primer orden}
  La cláusula $C$ es una \textbf{resolvente binaria} de las cláusulas $C_1$ y
  $C_2$ si existen una separación de variables $(\theta_1,\theta_2)$ de $C_1$ y
  $C_2$, un literal $L_1\in C_1$, un literal $L_2\in C_2$ y un UMG $\sigma $ de
  $L_1\sigma_1$ y $L_2^c\theta_2$ tales que
  $$C=(C_1\theta_1\sigma \backslash \{L_1\theta_2\sigma_1\})\cup (C_2\theta_2\sigma \backslash \{L_2\theta_2\sigma \} ) $$.

  Siendo nuestra implementación:

  \begin{code}
resolBin c1 c2 | s /= [] =
          renombramiento (c1'' +! c2'') (head s)
               | otherwise =
          error "No se puede unificar"
    where
      (c1', c2') = separacionVars c1 c2
      (p,q) = (litMisNom c1' c2')
      s = unifClau p q
      c1'' = borra p c1'
      c2'' = borra q c2'
\end{code}


Y un ejemplo sería:

\begin{sesion}
-- | Ejemplo
-- >>> let c1 = C [Neg (Atom "P" [tx]), Atom "Q" [Ter "f" [tx]]]
-- >>> let c2 = C [Neg (Atom "Q" [tx]), Atom "R" [Ter "g" [tx]]]
-- >>> resolBin c1 c2
-- \{¬P[x1],R[g[f[x1]]]\}
\end{sesion}
\end{block}

\end{column} % End of the first column

\begin{column}{\sepwid}\end{column} % Empty spacer column

\begin{column}{\twocolwid} % Begin a column which is two columns wide (column 2)

\begin{columns}[t,totalwidth=\twocolwid] % Split up the two columns wide column

\begin{column}{\onecolwid}\vspace{-.6in} % The first column within column 2 (column 2.1)

  \begin{alertblock}{Correspondencia de Curry-Howard}
    Se establece la siguiente correspondencia entre matemáticas y programación:
    \begin{itemize}
    \item Las proposiciones son tipos.
    \item Las demostraciones son programas.
    \end{itemize}
  \end{alertblock}

  \begin{block}{Demostraciones gratis}
    Debido a lo que establece la correspondencia podemos analizar
    proposiciones de la siguiente manera. Dada una proposición
    podemos expresarla mediante un tipo adecuado, y una vez que obtenemos
    el tipo, la existencia de una función con ese tipo de dato establece
    que la proposición es cierta, es decir, es equivalente a dar una
    demostración de la misma.
    
  \end{block}
  
%----------------------------------------------------------------------------------------

\end{column} % End of column 2.1

\begin{column}{\onecolwid}\vspace{-.6in} % The second column within column 2 (column 2.2)

% ---------------------------------------------------------------------------------------
% Ejemplos
%---------------------------------------------------------------------------------------

  \begin{block}{Ejemplos de la correspondencia}
    \begin{itemize}
    \item La proposición $(a \rightarrow a)$ vendría demostrada por:
    \begin{code}
    identidad :: a -> a
    identidad x = x
  \end{code}
\item  La implicación $(a \rightarrow b \rightarrow a \wedge b)$ no es más que:
  \begin{code}
 (,) :: a -> b -> (a, b)
 (,) x y = (x,y) 
\end{code}
\item La regla $(a \wedge b \rightarrow a)$ se vería demostrada por:
  \begin{code}
fst :: (a,b) -> a
fst (x,y) = x
\end{code}
\item la siguiente proposición $((a \rightarrow b \rightarrow c) \rightarrow (b \rightarrow a \rightarrow c) )$ es:
  \begin{code}
flip :: (a -> b -> c) -> b -> a -> c
flip f x y = f y x  
\end{code}

    \end{itemize}
  \end{block}
  

\end{column} % End of column 2.2

\end{columns} % End of the split of column 2 - any content after this will now take up 2 columns width


%----------------------------------------------------------------------------------------
%	Centro
%----------------------------------------------------------------------------------------
\begin{alertblock}{Conclusión}
  Matemáticas y programación son dos caras de una misma moneda.
\end{alertblock} 
%----------------------------------------------------------------------------------------

  


\begin{columns}[t,totalwidth=\twocolwid] % Split up the two columns wide column again

\begin{column}{\onecolwid} % The first column within column 2 (column 2.1)

  \begin{block}{Usando github}
    Para el desarrollo del trabajo se ha empleado github. Github permite:
    \begin{itemize}
    \item Coordinación dinámica entre tutor y alumno.
    \item Control de versiones.
    \item Manejo eficiente de un proyecto.
    \item Aprendizaje de \texttt{git}.
    \end{itemize}
    
  \end{block}


%----------------------------------------------------------------------------------------

\end{column} % End of column 2.1


\begin{column}{\onecolwid} % The second column within column 2 (column 2.2)

  \begin{block}{Usando Doctest}
    Para la prueba de los ejemplos que se han ido planteando
    ante cada implementación a lo largo del trabajo se ha empleado
    \texttt{doctest}. Permite localizar errores, así como
    hacer comprobaciones de coherencia entre implementaciones
    de distintos módulos.
  \end{block}
  
%----------------------------------------------------------------------------------------

\end{column} % End of column 2.2

\end{columns} % End of the split of column 2

\end{column} % End of the second column

\begin{column}{\sepwid}\end{column} % Empty spacer column

\begin{column}{\onecolwid} % The third column

% -------------------------------------------------------------------------------------
%  HERBRAND
% -----------------------------------------------------------------------------
  
%----------------------------------------------------------------------------------------
%	REFERENCias
%----------------------------------------------------------------------------------------

\begin{block}{Referencias}

\nocite{*} % Insert publications even if they are not cited in the poster
\small{\bibliographystyle{unsrt}
\bibliography{TFG1}\vspace{0.75in}}
\end{block}


%----------------------------------------------------------------------------------------

\end{column} % End of the third column

\end{columns} % End of all the columns in the poster

\end{frame}







% End of the enclosing frame





\end{document}

%%% Local Variables:
%%% mode: latex
%%% TeX-master: t
%%% End:
