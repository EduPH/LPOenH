\documentclass{beamer}

\usepackage[utf8x]{inputenc}       % Acentos de UTF8
\usepackage[spanish]{babel}        % Castellanización.
\usepackage[T1]{fontenc}           % Codificación T1 con European Computer
                                   % Modern.  
\usepackage{graphicx}
\usepackage{fancyvrb}              % Verbatim extendido
\usepackage{makeidx}               % Índice
\usepackage{amsmath}               % AMS LaTeX
\usepackage{amsthm} 
\usepackage{latexsym}
\usepackage{mathpazo}              % Fuentes semejante a palatino
\usepackage[scaled=.90]{helvet}
\usepackage{cmtt}
\usepackage{xfrac}
\renewcommand*\familydefault{\ttdefault} %% Only if the base font of the document is to be typewriter style
\usepackage{tkz-berge}
\usetikzlibrary{shapes,trees}
\usepackage{courier}

\usetheme{Frankfurt}
\usecolortheme{dove}
% Template
\useinnertheme{rectangles} %rectangle bullets etc
\beamertemplatenavigationsymbolsempty	%no navigation bar
\setbeamercovered{transparent}		%future bullet points transparent 
\setbeamertemplate{frametitle}[default][colsep=-4bp,rounded=false,shadow=false] %no shadows
\definecolor{dark-gray}{gray}{0.80} %color for the navigation squares


% Pie
\setbeamercolor{section in head/foot}{fg=black, bg=white}
\makeatletter
\setbeamertemplate{footline}
{
  \leavevmode%
  \hbox{%
  \begin{beamercolorbox}[wd=.333333\paperwidth,ht=2.25ex,dp=1ex,center]{section in head/foot}%
    \usebeamerfont{author in head/foot}\insertshortauthor~~\beamer@ifempty{\insertshortinstitute}{}{(\insertshortinstitute)}
  \end{beamercolorbox}%
  \begin{beamercolorbox}[wd=.333333\paperwidth,ht=2.25ex,dp=1ex,center]{section in head/foot}%
    \usebeamerfont{title in head/foot}\insertshorttitle
  \end{beamercolorbox}%
  \begin{beamercolorbox}[wd=.333333\paperwidth,ht=2.25ex,dp=1ex,right]{section in head/foot}%
    \usebeamerfont{date in head/foot}\insertshortdate{}\hspace*{2em}
    \insertframenumber{} / \inserttotalframenumber\hspace*{2ex} 
  \end{beamercolorbox}}%
  \vskip0pt%
}

%

\setbeamertemplate{mini frame}
{%
  \begin{pgfpicture}{0pt}{0pt}{.1cm}{.1cm}
    \pgfpathrectangle{\pgfpointorigin}{\pgfpoint{\the\beamer@boxsize}{\the\beamer@boxsize}}
    \pgfusepath{fill,stroke}
  \end{pgfpicture}%
}

\def\slideentry#1#2#3#4#5#6{%
  %section number, subsection number, slide number, first/last frame, page number, part number
  \ifnum#6=\c@part\ifnum#2>0\ifnum#3>0%
    \ifbeamer@compress%
      \advance\beamer@xpos by1\relax%
    \else%
      \beamer@xpos=#3\relax%
      \beamer@ypos=#2\relax%
    \fi%
  \hbox to 2pt{%
    \beamer@tempdim=-\beamer@vboxoffset%
    \advance\beamer@tempdim by-\beamer@boxsize%
    \multiply\beamer@tempdim by\beamer@ypos%
    \advance\beamer@tempdim by -.05cm%
    \raise\beamer@tempdim\hbox{%
      \beamer@tempdim=\beamer@boxsize%
      \multiply\beamer@tempdim by\beamer@xpos%
      \advance\beamer@tempdim by -\beamer@boxsize%
      \advance\beamer@tempdim by 1pt%
      \kern\beamer@tempdim
      \global\beamer@section@min@dim\beamer@tempdim
      \hbox{\beamer@link(#4){%
          \usebeamerfont{mini frame}%
          \ifnum\c@section>#1%
            \color{dark-gray}%
          \else%
            \ifnum\c@section=#1%
              \ifnum\c@subsection>#2%
                \color{dark-gray}%
              \else%
                \ifnum\c@subsection=#2%
                  \ifnum\c@subsectionslide>#3%
                    \color{dark-gray}%
                  \else%
                    \color{fg}%
                  \fi%
                \else%
                  \color{dark-gray}%
                \fi%
              \fi%
            \else%
              \color{dark-gray}%
            \fi%
          \fi%
          \usebeamertemplate{mini frame}%
        }}}\hskip-10cm plus 1fil%
  }\fi\fi%
  \else%
  \fakeslideentry{#1}{#2}{#3}{#4}{#5}{#6}%
  \fi\ignorespaces
  }

  %

  
\def\pdftex@driver{pdftex.def}
\ifx\Gin@driver\pdftex@driver
  \def\pgfsys@color@unstacked#1{%
    \pdfliteral{\csname\string\color@#1\endcsname}%
  }
\fi



\input definiciones

%Information to be included in the title page:
\title[LPO en Haskell]{Lógica de primer orden en Haskell}

\titlegraphic{\includegraphics[width=2cm]{sello}}
\author[Eduardo P.]{Eduardo Paluzo Hidalgo}
\institute[US]{Universidad de Sevilla}
\date{\today}


\begin{document}
 
\frame{\titlepage}

\frame{\tableofcontents}
\section{Introducción}
\begin{frame}
  \frametitle{Introducción}
  La realización de este trabajo ha supuesto:
  \begin{itemize}
  \item Implementación de la lógica de primer orden en Haskell.
  \item Implementación de algoritmos propios de la lógica de primer orden.
  \item Relación entre matemáticas y programación.
  \item Aprendizaje de recursos como git y doctest.
  \item Empleo de librerías de Haskell.
  \end{itemize}
\end{frame}


\begin{frame}
  Haskell tiene las siguientes características:

  \begin{itemize}
  \item Es un lenguaje de programación de tipo estático, es decir, el tipo de una variable
    se conoce en el momento de la compilación. Lleva a la necesidad de especificar los tipos. 
  \item Infiere los tipos.
  \item Es puramente funcional, es decir, \textbf{toda función en Haskell es una función en el sentido matemático}
  \item Es perezoso, es decir, las funciones no evalúan sus argumentos.
  \item Tiene una amplia gama de paquetes ya que las contribuciones ``Open source'' son muy activas. 
  \end{itemize}
\end{frame}

\begin{frame}
  La lógica de primer orden surge como extensión de la lógica proposicional ante carencias que ésta presenta. Es decir, se trata
  de una extensión de la lógica proposicional persiguiendo mayor fidelidad en el objetivo que la lógica proposicional
  persigue, modelar el razonamiento y aportar una manera de formalización.
  
\end{frame}
\section{Sintaxis y semántica}

\subsection{Implementación de la lógica}
\begin{frame}
  El \textbf{lenguaje de la lógica de primer orden} está compuesto por

\begin{enumerate}
\item Variables proposicionales $p,q,r,\dots$
\item Conectivas lógicas:
  \begin{center}
   \begin{tabular}{| l | l |}
     \hline
      $\neg$            & \text{Negación} \\ \hline
      $\vee$            & \text{Disyunción} \\ \hline
      $\wedge$          & \text{Conjunción} \\ \hline
      $\rightarrow$     & \text{Condicional} \\ \hline
      $\leftrightarrow$ & \text{Bicondicional}\\
     \hline
   \end{tabular}
 \end{center}
\item Símbolos auxiliares $"(",")"$
\item Cuantificadores: $\forall$ (Universal) y $\exists$ (Existencial)
\item Símbolo de igualdad: $=$
\item Constantes: $a,b,\dots,a_1,a_2,\dots$
\item Símbolos de relación: $P,Q,R,\dots $
\item Símbolos de función: $f,g,h\dots $
\end{enumerate}
\end{frame}
\begin{frame}[fragile]

  Implementamos las fórmulas
  
  \frametitle{Sintaxis y semántica}
  \begin{code}
    data Form = Atom Nombre [Termino]
          | Ig Termino Termino
          | Neg Form
          | Impl Form Form
          | Equiv Form Form
          | Conj [Form]
          | Disy [Form]
          | PTodo Variable Form
          | Ex Variable Form
  deriving (Eq,Ord,Generic)
  \end{code}
\end{frame}

\begin{frame}[fragile]
  Ejemplos de fórmulas:

  \vspace{5mm}
\begin{sesion}
> Disy [PTodo x (Atom "P" [tx]),Ex y (Atom "Q" [ty])]
(∀x P[x]⋁∃y Q[y])


> Neg (Ex x (Impl (Atom "P" [tx]) (PTodo x (Atom "P" [tx]))))
¬∃x (P[x]⟹∀x P[x])
\end{sesion}
\end{frame}

\begin{frame}
 Una fórmula es susceptible de ser interpretada, lo que permite que dependiendo de la interpretación, su evaluación puede ser verdadera o falsa. 
\end{frame}

\begin{frame}[fragile]
\begin{code}
type Interpretacion a =
      (InterpretacionR a, InterpretacionF a)  
\end{code}

\begin{code}
valorF :: Eq a => Universo a -> Interpretacion a
             -> Asignacion a -> Form -> Bool
valorF u (iR,iF) a (Atom r ts) =   
    iR r (map (valorT iF a) ts)
valorF u (_,iF) a (Ig t1 t2)   =   
    valorT iF a t1 == valorT iF a t2
valorF u i a (Neg g)           =  
    not (valorF u i a g)
valorF u i a (Impl f1 f2)      =  
    valorF u i a f1 <= valorF u i a f2
valorF u i a (Equiv f1 f2)     =  
    valorF u i a f1 == valorF u i a f2
\end{code}  
\end{frame}


\begin{frame}[fragile]
\begin{code}
valorF u i a (Conj fs)   =
    all (valorF u i a) fs
valorF u i a (Disy fs)   =  
    any (valorF u i a) fs
valorF u i a (PTodo v g) = 
    and [valorF u i (sustituye a v d) g | d <- u]
valorF u i a (Ex v g)    = 
    or  [valorF u i (sustituye a v d) g | d <- u]    
\end{code}
\end{frame}

\begin{frame}[fragile]
Veamos un ejemplo:
\begin{code}  
asignacion :: Variable -> Int
asignacion _ = 0
interpretacionF :: String -> [Int] -> Int
interpretacionF "cero" []    = 0
interpretacionF "s"    [i]   = succ i
interpretacionF "mas"  [i,j] = i + j
interpretacionF "por"  [i,j] = i * j
interpretacionF _ _          = 0
interpretacionR :: String -> [Int] -> Bool
interpretacionR "R" [x,y] = x < y  
interpretacionR _ _       = False
interpretacion = (interpretacionR,interpretacionF)
formula = Ex x (Atom "R" [cero,tx])
\end{code}
\begin{sesion}
  
> valorF [0..] interpretacion asignacion formula
True
\end{sesion}
\end{frame}
\section{El método de los tableros semánticos}
\subsection{Implementaciones previas}
\begin{frame}
  El \textbf{método de los tableros semánticos} tiene como objetivo determinar si una fórmula es consistente, así como la búsqueda de modelos. Para su implementación es necesaria:
  \begin{itemize}
  \item Sustitución: aplicación que asocia términos a variables.
  \item Unificación: una sustitución $S$ de dos términos $t_1$ y $t_2$ tal que $S(t_1)=S(t_2)$. 
  \item Formas normales $\longrightarrow$ forma de Skolem.
  \end{itemize}
\underline{Teorema}: El cálculo de tableros semánticos es \textbf{adecuado} y \textbf{completo}.  
\end{frame}

\begin{frame}[fragile]
\begin{center}
  Fórmulas alfa
\vspace{5mm}


   $\begin{array}{|l|l| } \hline
     \neg (F_1 \rightarrow F_2) & F_1 \wedge F_2               \\ \hline
     \neg(F_1 \vee F_2)         & F_1 \wedge \neg F_2          \\ \hline
     F_1 \leftrightarrow F_2    & (F_1 \rightarrow F_2) \wedge 
                                  (F_2 \rightarrow F_1)        \\ \hline
   \end{array}$
   \vspace{5mm}

Fórmulas beta

\vspace{5mm}
   $\begin{array}{|l|l|} \hline
    F_1 \rightarrow F_2             & \neg F_1 \vee F_2                \\ \hline
    \neg (F_1 \wedge F_2)           & \neg F_1 \vee \neg F_2           \\ \hline
    \neg (F_1 \leftrightarrow F_2)  & \neg (F_1 \rightarrow F_2 ) \vee 
                                      (\neg F_2 \rightarrow F_1)       \\ \hline
    \end{array}$

    \vspace{3mm}
    Fórmulas gamma \quad
    $\begin{array}{|l|c|} \hline
    \forall x F      & F [x/t]       \\ \hline
    \neg \exists x F & \neg F [x/t ] \\ \hline
     \end{array}$
     \vspace{3mm}

     Fórmulas delta \quad 
      $\begin{array}{|l|l|} \hline
    \exists x F    & F[x / a]       \\ \hline
    \neg \forall F & \neg F [x / a] \\ \hline
  \end{array}$
    
\end{center}    
\end{frame}

\subsection{Algoritmo}
\begin{frame}
\begin{itemize}
\item Sea $\mathcal{T}$ un tablero de $S$ y $S_1$ la etiqueta de
  una hoja  de $\mathcal{T}$.
  \begin{enumerate}
  \item Si $S_1$ contiene una fórmula y su negación, entonces el árbol
    obtenido añadiendo como hijo de $S_1$ el nodo etiquetado con
    $\left\{ \perp \right\}$ es un tablero de $S$.
  \item Si $S_1$ contiene una doble negación $\neg \neg F$, entonces
    el árbol obtenido añadiendo como hijo de $S_1$ el nodo etiquetado
    con $(S_1 \backslash \left\{\neg \neg F \right\})\cup \left\{ F \right\}$
    es un tablero de $S$.
  \item Si $S_1$ contiene una fórmula alfa $F$ de componentes $F_1$ y $F_2$,
    entonces el árbol obtenido añadiendo como hijo de $S_1$ el nodo etiquetado
    con $(S_1 \backslash \left\{ F \right\})\cup \left\{ F_1,F_2 \right\}$
    es un tablero de $S$.
  \item Si $S_1$ contiene una fórmula beta de $F$ de componentes $F_1$ y $F_2$,
    entonces el árbol obtenido añadiendo como hijos de $S_1$ los nodos
    etiquetados con $(S_1 \backslash \left\{ F \right\}) \cup \left\{ F_1 \right\}$ y
    $(S_1 \backslash \left\{ F \right\})\cup \left\{ F_2 \right\}$ es un tablero
    de $S$.
  \end{enumerate}
\end{itemize}  
\end{frame}

\begin{frame}[fragile]
\begin{code}
ramificacionP :: Int -> Nodo -> (Int,Tablero)
ramificacionP k nodo@(Nd i pos neg []) = (k,[nodo])
ramificacionP k (Nd i pos neg (f:fs))
  | atomo    f = 
      if literalATer f `elem` neg
      then (k,[])
      else (k,[Nd i (literalATer f : pos) neg fs])
  | negAtomo f = 
      if literalATer f `elem` neg
      then (k,[]) 
      else (k,[Nd i pos (literalATer f : neg) fs])
  | dobleNeg f = 
      (k,[Nd i pos neg (componentes f ++ fs)])
  | alfa     f = 
      (k,[Nd i pos neg (componentes f ++ fs)])
\end{code}  
\end{frame}

\begin{frame}[fragile]
  \begin{code}
ramificacionP k (Nd i pos neg (f:fs))
    | beta     f = 
        (k,[Nd (i++[n]) pos neg (f':fs)
           | (f',n) <- zip (componentes f) [0..] ])
    | gamma    f = (k-1, [Nd i pos neg (f':(fs++[f]))])
    where 
      (xs,g) = descomponer f
      b      =
          [(Variable nombre j, Var (Variable nombre i))
          | (Variable nombre j) <- xs]
      f'     = sustitucionForm b g
  \end{code}
\end{frame}

\subsection{Aplicación del algoritmo}

\begin{frame}[fragile]
  La implementación del método de tableros permite definir la función \texttt{(refuta k f)}
  que intenta refutar la fórmula \texttt{f} a profundidad \texttt{k}.
  \begin{code}
refuta :: Int -> Form -> Bool
refuta k f = esCerrado tab /= []
    where tab = expandeTablero k (tableroInicial f)  
  \end{code}
  Además, también se pueden implementar las siguientes funciones para determinar si una fórmula
  es un teorema y su satisfacibilidad. 
  \begin{code}
esTeorema, satisfacible :: Int -> Form -> Bool
esTeorema n = refuta n . skolem . Neg
satisfacible n = not . refuta n . skolem
\end{code}
\end{frame}

\begin{frame}[fragile]
  Por ejemplo:

\begin{code}  
> formula1
(P[x]⋁¬P[x])
> esTeorema 1 formula1
True
> formula2
∀x ∀y (R[x,y]⟹∃z (R[x,z]⋀R[z,y]))
> esTeorema 20 formula2
False
\end{code}  
\end{frame}
\section{Modelos de Herbrand}

\begin{frame}
  \textbf{Modelos de Herbrand:}
  \begin{itemize}
  \item Método constructivo para generar interpretaciones de fórmulas o conjuntos de ellas.
  \item Sirve para determinar la consistencia de un conjunto de cláusulas. (Tener un modelo de
    Herbrand es equivalente a ser consistente)
  \end{itemize}
\end{frame}

\begin{frame}[fragile]
  El \textbf{universo de Herbrand} de $F$ es el conjunto de términos básicos.
\begin{sesion}
> f3
(R[f[x]]⟹Q[a,g[b]])
> universoHerbrandForm 2 f3
[a,b,f[a],f[b],g[a],g[b],f[f[a]],f[f[b]],f[g[a]],
f[g[b]],g[f[a]],g[f[b]],g[g[a]],g[g[b]]]
\end{sesion}

\underline{Proposición:}
  \begin{equation*}
    H_0(L)= \left\{
      \begin{array}{ll}
        C, & \text{si } C \neq \emptyset \\
        {a}, & \text{en caso contrario (a nueva constante)} 
      \end{array} \right.
  \end{equation*}
  
  $H_{i+1}(L) = H_i(L)\cup \{f(t_1,\dots,t_n):f\in \mathcal{F}_n \text{ y } t_i\in H_i (L)\}$
\end{frame}

\begin{frame}[fragile]
La \textbf{base de Herbrand} es el conjunto de átomos básicos.

\begin{sesion}
> f1
P[f[x]]
> baseHerbrandForm 2 f1
[P[a],P[f[a]],P[f[f[a]]]]  
\end{sesion}
\end{frame}

\begin{frame}[fragile]
  \textbf{Modelos de Herbrand:}

  \vspace{5mm}
\begin{code}
modelosHForm :: Int -> Form -> [AtomosH]
modelosHForm n f =
 [fs | fs <- subsequences bH
      , valorH uH (interpretacionH fs) f]
  where uH = universoHerbrandForm n f
        bH = baseHerbrandForm n f
\end{code}  
  
\end{frame}

\section{Resolución en la lógica de primer orden}
\begin{frame}
  \textbf{Resolución:}

  \begin{itemize}
  \item Forma clausal.
  \item Resolventes binarias en lógica proposicional y lógica de primer orden.
  \end{itemize}
  La resolución aporta reglas para obtener deducciones lógicas a partir de cláusulas.

  Se pueden implementar diversos algoritmos que aplican resolución.
\end{frame}

\begin{frame}[fragile]
\begin{code}
resolBin c1 c2 
    | s /= [] = renombramiento (c1'' +! c2'') (head s)
    | otherwise = error "No se puede unificar"
    where
      (c1', c2') = separacionVars c1 c2
      (p,q) = (litMisNom c1' c2')
      s = unifClau p q
      c1'' = borra p c1'
      c2'' = borra q c2'
\end{code}  
\end{frame}

\begin{frame}[fragile]
\begin{sesion}
> let c1 = C [Neg (Atom "P" [tx]), Atom "Q" [Ter "f" [tx]]]
> let c2 = C [Neg (Atom "Q" [tx]), Atom "R" [Ter "g" [tx]]]
> c1
\{¬P[x],Q[f[x]]\}
> c2
\{¬Q[x],R[g[x]]\}
> resolBin c1 c2
\{¬P[x1],R[g[f[x1]]]\}
\end{sesion}

\end{frame}
\section{Correspondencia de Curry-Howard}
\begin{frame}
  \textbf{Correspondencia de Curry-Howard:}
\vspace{5mm}
  \begin{itemize}
  \item Las proposiciones son tipos.
  \item Las demostraciones son funciones.
  \end{itemize}
\end{frame}


\subsection{Ejemplos}
\begin{frame}[fragile]  
  $$a \rightarrow a$$
\pause
\begin{sesion}
a -> a
\end{sesion}
\pause
\begin{code}
identidad :: a-> a
identidad x = x
\end{code}  

\end{frame}

\begin{frame}[fragile]
  $$a \rightarrow b \rightarrow a\wedge b $$
\pause

\begin{sesion}
a -> b -> (a,b)
\end{sesion}
\pause

\begin{code}
(,) :: a -> b -> (a,b)
(,) x y = (x,y)
\end{code}

\end{frame}

\begin{frame}[fragile]
  $$a \wedge b \rightarrow a $$
\pause

\begin{sesion}  
(a,b) -> a
\end{sesion}

\pause

\begin{code}
fst :: (a,b) -> a
fst (x,y) = x
\end{code}

\end{frame}

\begin{frame}[fragile]
  $$a \rightarrow a\wedge b $$

\pause
\begin{sesion}
a -> (a,b)
\end{sesion}
  
\pause
No existe ninguna función.
\end{frame}

\begin{frame}[fragile]
  $$((a\wedge b) \rightarrow c) \rightarrow (a \rightarrow b \rightarrow c) $$

  \pause
  
\begin{code}
curry :: ((a, b) -> c) -> a -> b -> c  
\end{code}
\end{frame}


\begin{frame}[fragile]
  $$a \rightarrow (a \vee b)$$

  \pause

\begin{sesion}
a -> Either a b  
\end{sesion}

\pause

\begin{code}
introDisy :: a -> Either a b
introDisy x = Left x  
\end{code}
  
\end{frame}
\begin{frame}[fragile]
  $$(a\rightarrow c) \vee (b\rightarrow c) \rightarrow a \rightarrow b \rightarrow c  $$

  \pause
  
\begin{code}
fun :: Either (a -> c) (b -> c) -> a -> b -> c
fun (Left f)  x y = f x
fun (Right g) x y = g y
\end{code}
\end{frame}

\begin{frame}
\begin{table}[htbp]
\begin{center}
\begin{tabular}{|l|l|}
\hline
Haskell & Lógica \\
\hline \hline
Variables de tipo:  \texttt{a} & V. proposicionales: $p$  \\ \hline
Tipos & Proposiciones \\ \hline
Tipos de funciones: \texttt{a -> b} & Implicaciones: $p\rightarrow q$ \\ \hline
Tuplas: \texttt{(a,b)} & Conjunciones : $p\wedge q$ \\ \hline
Either: \texttt{Either a b} & Disyunciones: $p\vee q$ \\ \hline  
Tipo habitado & Demostración \\ \hline  
\end{tabular}
\end{center}
\end{table}
  

\end{frame}
\section{Bibliografía}
\begin{frame}[allowframebreaks]
\nocite{*} % Insert publications even if they are not cited in the poster
\small{\bibliographystyle{unsrt}
\bibliography{TFG}\vspace{0.75in}}
\end{frame}
\end{document}
%%% Local Variables:
%%% mode: latex
%%% TeX-master: t
%%% End:
